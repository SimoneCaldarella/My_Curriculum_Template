%%%%%%%%%%%%%%%%%%%%%%%%%%%%%%%%%%%%%%%
% PlushCV - One Page Two Column Resume
% LaTeX Template
% Version 1.0 (11/28/2021)
%
% Author:
% Shubham Mazumder (http://mazumder.me)
%
% Hacked together from:
% https://github.com/deedydas/Deedy-Resume
%
% IMPORTANT: THIS TEMPLATE NEEDS TO BE COMPILED WITH XeLaTeX
%
% 
%%%%%%%%%%%%%%%%%%%%%%%%%%%%%%%%%%%%%%
% 
% TODO:
% 1. Figure out a smoother way for the document to flow onto the next page.
% 3. Add more icon options 
% 4. Fix hacky left alignment on contact line
% 5. Remove Hacky fix for awkward extra vertical space
% 
%%%%%%%%%%%%%%%%%%%%%%%%%%%%%%%%%%%%%%
%
% CHANGELOG:
%
%%%%%%%%%%%%%%%%%%%%%%%%%%%%%%%%%%%%%%%
%
% Known Issues:
% 1. Overflows onto second page if any column's contents are more than the vertical limit.
%%%%%%%%%%%%%%%%%%%%%%%%%%%%%%%%%%%%%%
%%Icons:
%%Main: https://icons8.com/icons/carbon-copy
%%%%%%%%%%%%%%%%%%%%%%%%%%%%%%%%%%

\documentclass[]{plushcv}
\usepackage{fancyhdr}
\pagestyle{fancy}
\fancyhf{}
\begin{document}

%%%%%%%%%%%%%%%%%%%%%%%%%%%%%%%%%%%%%%
%
%     TITLE NAME
%
%%%%%%%%%%%%%%%%%%%%%%%%%%%%%%%%%%%%%%
\namesection{Simone}{Caldarella}{Student Intern @ SISLab}{\contactline{\href{https://www.github.com/simonecaldarella}{SimoneCaldarella}}{\href{https://www.linkedin.com/in/simone-caldarella-16930b16a/}{SC}}{\href{mailto:simone.caldarella98@gmail.com}{simone.caldarella98@gmail.com}}{\href{tel:+39 3884495671}{+39 3884495671}}}

% \namesection{Firstname}{Lastname}{Full Stack Software Engineer}{\contactline{\href{https://www.mazumder.me}{mazumder.me}}{\href{https://www.github.com/sansquoi}{sansquoi}}{\href{https://www.linkedin.com/mazumders}{mazumders}}{\href{mailto:shubham.mazumder@gmail.com}{first.last@email.com}}{\href{tel:+1999999999}{9999999999}}}

%%%%%%%%%%%%%%%%%%%%%%%%%%%%%%%%%%%%%%
%
%     COLUMN ONE
%
%%%%%%%%%%%%%%%%%%%%%%%%%%%%%%%%%%%%%%

\begin{minipage}[t]{0.70\textwidth} 


%%%%%%%%%%%%%%%%%%%%%%%%%%%%%%%%%%%%%%
%     EXPERIENCE
%%%%%%%%%%%%%%%%%%%%%%%%%%%%%%%%%%%%%%

\section{Education}
\runsubsection{Artificial Intelligence Systems}
\descript{| M.Sc. }
\location{October 2020 – Current | University of Trento, Italy}
\vspace{\topsep} % Hacky fix for awkward extra vertical space
\begin{tightemize}
\sectionsep
\item Language: English
\item Computer Vision specialization
\end{tightemize}
\sectionsep

\runsubsection{Information Engineering and Computer Science}
\descript{| B.Sc. }
\location{July 2017 – September 2020 | University of Brescia, Italy}
\begin{tightemize}
\sectionsep
\item Thesis: "DeepFake: a technological evolution with use cases between threats and opportunities"
\end{tightemize}
\sectionsep


%%%%%%%%%%%%%%%%%%%%%%%%%%%%%%%%%%%%%%
%     Projects
%%%%%%%%%%%%%%%%%%%%%%%%%%%%%%%%%%%%%%

\section{Projects}

\runsubsection{Open Domain Aspect Based Sentimet Analysis}
\descript{| NLP}
\location{Two-stage Target/Aspect Based Sentiment Analysis Model}
\begin{tightemize}
\item Target extraction model and Polarity classification model developed using Bert as backbone
\item Language and libraries: Python, Transformers, PyTorch, NumPy, Matplotlib
\item Github repository: \href{https://github.com/SimoneCaldarella/ABSA_Project}{ABSA}
\end{tightemize}
\sectionsep

\runsubsection{Person Attribute Recognition and Re-Identification}
\descript{| Deep Learning}
\location{Pipeline with a multi-label multi-class CNN for Person AR and a Siamese CNN for Person Re-ID}
\begin{tightemize}
\item Custom CNN models developed using ResNet18 as backbone, followed by custom classifiers and module. Techniques to deal with data unbalancing employed (Balanced Cross Entropy, Focal Loss)
\item Language and libraries: Python, PyTorch, TensorBoard, Matplotlib, NumPy, Matplotlib
\end{tightemize}
\sectionsep

\runsubsection{HandMouse}
\descript{| Image Processing}
\location{Hand tracking detection algorithm using laptpot's RGB camera}
\begin{tightemize}
\item Custom background substraction algorithm combined with "Good Features to tracks" as relevant hand keypoints and convex hull computation
\item Languages and libraries: Python, OpenCV, NumPy, Tkinter
\item Github repository:
\href{https://github.com/SimoneCaldarella/HandMouse}{HandMouse}
\end{tightemize}
\sectionsep

\runsubsection{RobotIEEE}
\descript{| App Development}
\location{Android app for communications between SumoRobot and a planning algorithm online, developed for IEEE Metrology for Industry 4.0 and IoT}
\begin{tightemize}
\item Android App developed to receive instructions from a server via http requests, parse them and send them to the SumoRobot via bluetooth protocol
\item Languages and libraries: Java, Android, AndroidStudio, Bluetooth, HTTPUrlConnection
\item Github repository:
\href{https://github.com/SimoneCaldarella/robotieeeApp/}{RobotIEEE}
\end{tightemize}
\sectionsep

\runsubsection{YouCantSeeMe}
\descript{| Low Level Programming}
\location{Steganography algorithm developed using only C for audio and images}
\begin{tightemize}
\item BMP images and WAV audio parsing algorithm developed to find the best bits in encoding to hide a secret message, without modifying aspect of the file.
\item Language and libraries: C
\item Github repository:
\href{https://github.com/SimoneCaldarella/YouCantSeeMe}{YouCantSeeMe}
\end{tightemize}
\sectionsep


%%%%%%%%%%%%%%%%%%%%%%%%%%%%%%%%%%%%%%
%     AWARDS
%%%%%%%%%%%%%%%%%%%%%%%%%%%%%%%%%%%%%%

% \section{Awards} 
% \begin{tabular}{rll}
% 2020	     & Finalist & Lorem Ipsum\\
% 2018	     & $2^{nd}$ & Dolor Sit Amet\\
% 2015	     & Finalist  & Cras posuere\\
% \\
% \end{tabular}
% \sectionsep
%%%%%%%%%%%%%%%%%%%%%%%%%%%%%%%%%%%%%%
%
%     COLUMN TWO
%
%%%%%%%%%%%%%%%%%%%%%%%%%%%%%%%%%%%%%%

\end{minipage} 
\hfill
\begin{minipage}[t]{0.25\textwidth} 

\section{Achievements} 
\subsection{Chairman}
\descript{Chairmanship in IEEE Student Branch of Brescia}
\location{2018-2019}
\sectionsep

\subsection{Judge and Mentor}
\descript{Judge and Mentor of Nasa Space Apps challenge Brescia}
\location{2019}
\sectionsep

\subsection{IEEEXtreme 1° Place in Italy }
\descript{1° Place in Italy at International Hackathon organised by IEEE}
\location{2018}
\sectionsep

\section{Experience} 
\subsection{Internship @ SISLab}
\descript{University of Trento | DISI}
\location{Jan 2022 - Present}
Development of Dialogue Responses Generation models

\section{Certificate} 
\subsection{IELTS - {\normalfont 6.5 | 9/2020}}
\sectionsep


%%%%%%%%%%%%%%%%%%%%%%%%%%%%%%%%%%%%%%
%     SKILLS
%%%%%%%%%%%%%%%%%%%%%%%%%%%%%%%%%%%%%%

\section{Skills}
\subsection{Programming}
\sectionsep
Python \textbullet{} Java \textbullet{} C \textbullet{} \LaTeX \\ Shell \textbullet{} Matlab \\
\sectionsep
\sectionsep

\subsection{Libraries/Frameworks}
\sectionsep
PyTorch \textbullet{} HuggingFace \textbullet{} SpaCy \textbullet{} Matplotlib \\ OpenCV
\textbullet{} NumPy \\ 
\sectionsep
\sectionsep

\subsection{Tools/Platforms}
\sectionsep
Git \textbullet{} Github \textbullet{} Android Studio \\ Docker \textbullet{} Overleaf \textbullet{} Excel \\
\sectionsep
\sectionsep

\subsection{Fields of Interest}
\sectionsep
Deep Learning \textbullet{} Machine Learning \textbullet{} Computer Vision \\ NLP \textbullet{} Statistics \textbullet{} Calculus \\ Software Engineering
\sectionsep
\sectionsep

%%%%%%%%%%%%%%%%%%%%%%%%%%%%%%%%%%%%%%
%     EDUCATION
%%%%%%%%%%%%%%%%%%%%%%%%%%%%%%%%%%%%%%



% %%%%%%%%%%%%%%%%%%%%%%%%%%%%%%%%%%%%%%
% %     REFERENCES
% %%%%%%%%%%%%%%%%%%%%%%%%%%%%%%%%%%%%%%


%%%%%%%%%%%%%%%%%%%%%%%%%%%%%%%%%%%%%%
%     COURSEWORK
%%%%%%%%%%%%%%%%%%%%%%%%%%%%%%%%%%%%%%

% \section{Coursework}

% \subsection{Graduate}
% Graduate Algorithms \textbullet{}\\ 
% Advanced Computer Architecture \textbullet{}\\ 
% Operating Systems \textbullet{}\\ 
% Artificial Intelligence \textbullet{}\\
% Visualization For Scientific Data \\
% \sectionsep

% \subsection{Undergraduate}

% Database Management Systems \textbullet{}\\
% Object Oriented Analysis and Design \textbullet{}\\
% Artificial Intelligence and Expert Systems \textbullet{}\\
% Scripting Languages and Web Tech \textbullet{}\\
% Software Engineering \\

\end{minipage} 
\end{document}  \documentclass[]{article}

